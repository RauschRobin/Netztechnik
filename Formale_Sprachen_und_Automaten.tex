\documentclass[12pt,a4paper]{article}

\usepackage[utf8]{inputenc}
\usepackage[ngerman]{babel}
\usepackage[T1]{fontenc}
\usepackage{amsmath}
\usepackage{amsfonts}
\usepackage{amssymb}
\usepackage{graphicx}
\usepackage[left=2cm,right=2cm,top=2cm,bottom=2cm]{geometry}
\usepackage{multicol}
\usepackage{booktabs}
\usepackage[hidelinks]{hyperref}
\usepackage{tikz}
\usepackage{pgfplots}
\usepackage{blindtext}
\usepackage{array}
\usepackage{multirow}
\usepackage{bigdelim}
\usepackage{colortbl}
\usepackage{fancyhdr} 
\usepackage{tabularx}
\usepackage{xcolor}
\usepackage{color}
\usetikzlibrary{decorations.text}
\usetikzlibrary{tikzmark}
\pagestyle{fancy} 
	\fancyhf{} 
	\fancyhead[L]{\includegraphics[scale=0.05]{Bilder/dhbw.png}} 
	\fancyhead[C]{\slshape Formale Sprachen und Automaten} 
	\fancyhead[R]{\slshape LaTeX Version}
	\fancyfoot[C]{\thepage}
\usepackage{helvet}
\renewcommand{\familydefault}{\sfdefault}

\title{Formale Sprachen und Automaten}
\author{\slshape Robin Rausch}
\date{\slshape \today}
\begin{document}
\pagenumbering{Roman}
\maketitle
\tableofcontents
\newpage
\pagenumbering{arabic}
\section{Grundlagen}
\subsection{Alphabet}
Ein Alphabet $\varSigma$ ist eine nicht-leere Menge von Symbolen(Zeichen, Buchstaben).
Beispiel: $\varSigma_{ab} = { a, b }$

\subsection{Wort}
Ein Wort $w$ über dem Alphabet $\varSigma$(Sigma) ist eine endliche Folge von Symbolen aus $\varSigma$. Das Wort $w = abaabab$ wurde beispielsweise aus dem Alphabet $\varSigma_{ab}$ gebildet.\newline
Die Länge eines Wortes kann durch Betragsstriche angegeben werden. Beispiel: $|w| = 7$\newline
Ebenso kann man die Anzahl bestimmter Symbole in einem Wort bestimmen: $|w|_b = 3$\newline
Ein einzelnes Zeichen kann durch eckige Klammern angegeben werden: $w[2] = b$\newline
Wörter können bliebig konkateniert werden(hintereinanderschreiben ohne abstand): $w_1w_2 = abbabaab$ mit $w_1 = abba$ und $w_2 = baab$.\newline
Wörter dürfen auch potenziert werden: $w^3 = abaabababaabababaabab = www$\newline
Das leere Wort lautet $\epsilon$.

\subsection{Formale Sprachen}
Eine formale Sprache $L$ über einem Alphabet $\varSigma$ ist eine Menge von Wörtern aus $\varSigma^*: L \subseteq \varSigma^*$. Eine Sprache kann sowohl endlich als auch unendlich sein.\newline
Beispiel: $L_1 = \{w \in \varSigma_{bin}^* |$ $|w| \geqslant 2 \wedge w[|w| - 1] = 1\}$ ist die Menge aller Binärwörter, an deren vorletzter Stelle 1 steht.\newline
Das Produkt zweier formaler Sprachen: $L_1 \cdot L_2 = \{abac, abcb, bcac, bccb\}$ mit $L_1 = \{ab, bc\}$ und $L_2 = \{ ac, cb\}$.\newline
Sprachen können ebenfalls potenziert werden: $L^2 = \{ab, ba\} \cdot \{ab, ba\} = \{ abab, abba, baab, baba\}$

\subsection{Kleene Stern}
Für ein Alphabet $\varSigma$ und eine formale Sprache $L \subseteq \varSigma^*$ ist der Operator Kleene Stern wie folgt definiert: $L^* = \underset{n \in \mathbb{N}}{\bigcup} L^n$.\newline \newline
Beispiel: Sei $L_1 = \{ ab, ba\}$, dann $L^* = \{\epsilon, ab, ba, abab, abba, baab, baba, ababab, ...\}$.

\section{Reguläre Sprachen und endliche Ausdrücke}
\subsection{Reguläre Ausdrücke}
Ein regulärer Ausdruck über $\varSigma$ beschreibt eine formale Sprache.\newline
Die Menge aller regulären Ausdrücke über $\varSigma$ ist eine formale Sprache.\newline\newline
Beispiel: Sprache aller Wörter über $\varSigma_{abc}$, die nur aus genau zwei Symbolen bestehen:\newline
Ausdruck: $r_1 = (a + b + c)(a + b + c)$\newline
Sprache: $\mathcal{L}(r_1) = \{ w \in \varSigma_{abc}^*$ | $|w| = 2\}$\newpage
\noindent Operatoren:
\begin{center}
	\begin{figure}[!h]
		\includegraphics[width=\textwidth]{Bilder/RegulaereAusdruecke_Operatoren.PNG}
	\end{figure}
\end{center}
Nicht alle Operatoren sind für alle Typen zulässig:
\begin{center}
	\begin{figure}[!h]
		\includegraphics[width=\textwidth]{Bilder/Zulaessige_Operatoren.PNG}
	\end{figure}
\end{center}

\subsection{Endliche Automaten}
Endliche Automaten sind eine andere Darstellung einer regulären Sprache. Endliche Ausdrücke lassen sich in Reguläre Ausdrücke umformen. Genauso auch anders herum.\newline
Endliche Automaten lassen sich sowohl deterministisch als auch nicht-deterministisch darstellen.

\subsubsection{Deterministische endliche Automaten(DEA)}
Ein DEA hat endlich viele Zustände. Jeder mögliche Übergang muss hierbei behandelt werden können. D.h. für das Alphabet $\varSigma_{ab}$ muss von jedem Zustand sowohl ein $a$, als auch ein $b$ Übergang gegeben sein. Der Automat beginnt im Startzustand und muss im Endzustand enden. Wenn der Automat sich in einem Nicht-Endzustand befindet, befindet sich das Wort nicht in der Sprache, welche vom Automaten abgebildet wird.\newpage
\noindent Der DEA lässt sich durch folgendes 5-Tupel darstellen:\newline
$\mathcal{A} = (Q, \varSigma, \delta, q_0, F)$ mit den Komponenten:\newline
$Q$ ist eine endliche Menge von Zuständen\newline
$\varSigma$ ist ein endliches Alphabet\newline
$\delta: Q \times \varSigma \rightarrow Q$ ist die Übergangsfunktion\newline
$q_0 \in Q$ ist der Startzustand\newline
$F \subseteq Q$ ist die Menge der Endzustände\newline
\newline
Beispiel:\newline
\begin{center}
	\begin{figure}[!h]
		\includegraphics[width=\textwidth]{Bilder/DEA_Beispiel.PNG}
	\end{figure}
\end{center}

\subsubsection{Nicht-deterministische endliche Automaten(NEA)}

\subsubsection{Endliche Automaten und reguläre Ausdrücke}

\subsubsection{Minimierung}

\subsection{Nicht-reguläre Sprachen und das Pumping-Lemma}

\subsection{Eigenschaften regulärer Sprachen}

\section{Chomsky Grammatiken und kontextfreie Sprachen}

\section{Turing Maschine}

\section{Entscheidbarkeit}

\section{Berechenbarkeit}

\section{Komplexität}

\end{document}